%% LyX 2.2.3 created this file.  For more info, see http://www.lyx.org/.
%% Do not edit unless you really know what you are doing.
\documentclass[english]{article}
\usepackage[T1]{fontenc}
\usepackage[latin9]{inputenc}
\usepackage{amssymb}
\usepackage{graphicx}
\usepackage{esint}
\usepackage{babel}
\begin{document}

\title{Summer report}

\maketitle
I investigated how to simulate soft elastic object effectively using
only coarse meshes. We will break this investigation into 2 parts
in this report
\begin{enumerate}
\item How to quickly fit the material property of an object to a FEM simulation
through dynamics using DAC (\cite{Chen:2017:DNC:3072959.3073669}
Section 3).
\item Investigate FEM locking/numerical stiffening on coarse mesh. In particular,
I looked at the cause and how it affects the simulation in static
and dynamic.
\end{enumerate}

\section{Fitting Young's modulus using DAC\label{sec:Fitting-Young's-modulus}}

The material properties of elastic objects specified by the manufacturer
are often inaccurate (due to variation in fabrication process, I assume),
and calibration is needed for each object. Fitting material properties
through tracking is not an easy task. but DAC is a simple algorithm
and work surprisingly well in the cases considered in \cite{Chen:2017:DNC:3072959.3073669}.
DAC consists of the following steps:
\begin{enumerate}
\item Track the oscillation of a simple deformation mode of the object\label{enu:Track-the-oscillation}
\item Solve the harmonic inverse problem \cite{doi:10.1063/1.475324} from
the trajectory in \ref{enu:Track-the-oscillation}. and extract the
dominating frequency (frequency with largest amplitude), $\omega_{l}$
\label{enu:Solve-the-harmonic}
\item Create a mesh for the FEM simulation, with initial Young's modulus
$Y_{0}=1Pa$. Other material parameters (Poisson ratio, density, etc)
are not fitted here, so will be set to some physical values and stay
unchanged after DAC. For example Poisson ratio can be set to $P=0.48,$
and the density $\rho$ is set to the value specified by the manufacturer.
Also the material model can is arbitrary (linear, Neo-hookean etc)\label{enu:Create-a-mesh}
\item Find the lowest the eigenvalue problem\label{enu:Find-the-lowest}
\[
Kx=\lambda_{l}Mx,
\]
where $K$ and $M$ are the stiffness matrix and mass matrix from
the FEM simulation\footnote{If the model chosen in \ref{enu:Create-a-mesh}. is nonlinear, $K$
must be chosen at a equilibrium state to ensure positive definiteness.} \ref{enu:Create-a-mesh}., with the same boundary conditions from
the simple deformation in \ref{enu:Track-the-oscillation}.. 
\item Rescale the Young's modulus by \ref{enu:Solve-the-harmonic}. and
\ref{enu:Find-the-lowest}.\label{enu:Rescale-the-Young's}
\[
Y=Y_{0}\frac{\omega_{l}^{2}}{\lambda_{l}}
\]
\end{enumerate}
This fitting is based on the fact that the eigenvalues of the system
scale linearly with Young's modulus. After DAC, the Young's modulus
can be used for simulating the same object \emph{with the same mesh}.
The next section of this writeup investigate the mesh-dependent stiffness,
which prevent us from using DAC on the same object with different
mesh resolutions.

\section{Mesh-dependent stiffness}

Following the fact that Young's modulus scales linearly with the eigenvalue,
we observed that different mesh resolution of same object would have
different eigenvalues\footnote{As mesh gets finer and finer, the number of eigenvalue increases,
and the new eigenvalues corresponds to the high oscillatory modes
which damp out quickly. Here we are talking about lower eigenvalues,
which correspond to the dominating dynamic modes with larger amplitudes,
and should ideally stay the same across different mesh resolutions.}, meaning the apparent stiffness of the object is different, and the
modes would oscillate at different frequencies.

\subsection{A 2D Example}

First start from an polygon shape object (from Distmesh \cite{Persson:2004aa})
with 2 level of refinements (See Fig \ref{fig:Polygon-from-Distmesh},
\ref{fig:Coarse-mesh}, \ref{fig:Fine-mesh}). Solve the general eigenvalue
problem with Neumann B.C. for both mesh
\[
Kx=\lambda Mx
\]
and eliminate the null space corresponding to the rigid motion, we
plot the lowest eigenmode in Fig \ref{fig:Coarse-mesh-first} and
Fig \ref{fig:Fine-mesh-first}. These eigenmodes look similar on the
object, but the eigenvalues are very different, $\lambda=2.35$ for
the low resolution mesh, and $\lambda=1.99$ for the high resolution
mesh, meaning simulating on the low resolution mesh would make the
object look stiffer and oscillate faster. 

\begin{figure}
\includegraphics[scale=0.2]{report_images/polygon}

\caption{Polygon from Distmesh \cite{Persson:2004aa}\label{fig:Polygon-from-Distmesh}}

\end{figure}

\begin{figure}
\includegraphics[scale=0.35]{report_images/polygon_coarse}

\caption{Coarse mesh\label{fig:Coarse-mesh}}

\end{figure}

\begin{figure}
\includegraphics[scale=0.35]{report_images/polygon_fine}

\caption{Fine mesh\label{fig:Fine-mesh}}

\end{figure}

\begin{figure}
\includegraphics[scale=0.2]{report_images/polygon_coarse_firstmode}

\caption{Coarse mesh first mode with eigenvalue \label{fig:Coarse-mesh-first}$=2.3533$}
\end{figure}

\begin{figure}
\includegraphics[scale=0.2]{report_images/polygon_fine_firstmode}

\caption{Fine mesh first mode with eigenvalue\label{fig:Fine-mesh-first} $=1.99$}

\end{figure}

\subsection{Theoretical background\label{subsec:Theoretical-background}}

This discrepancy comes from the discretization error from FEM, so
it is safe to assume that the material will look softer and softer
as we refine the mesh. This is indeed the case, as indicated in \cite{Ciarlet:1968:NMH:2716620.2716941}
(lemma 3), where they showed that the approximate the general eigenvalue
problem of an elliptical B.V.P. with Ritz approximation/Galerkin method
will always lead to larger eigenvalues. Intuitively, this comes from
the fact that we are searching in a less inclusive function space
as we coarsen the mesh in the weak form of the elliptical B.V.P.

\subsection{A 1D example}

In this subsection we discuss a 1D example in more detail and demonstrate
DAC and it's drawback.

\subsubsection{Continuous case}

Consider the 1D wave equation with Dirichlet BC over an interval $x\in\left[0,L\right]$
\begin{eqnarray}
u_{tt} & = & c^{2}\nabla^{2}u\label{eq:wave eq}\\
u(t,x=0) & = & u(t,x=L)=0\nonumber 
\end{eqnarray}

Look for the solution of the form
\[
u(t,x)=e^{i\sqrt{\lambda}t}U(x)
\]
is equivalent to solve the eigenvalue problem for the operator
\begin{equation}
\mathbb{L}U=-c^{2}\nabla^{2}U=\lambda U\label{eq:eigenvalue prob}
\end{equation}
The analytical solution for the eigenvalues: 
\begin{equation}
\lambda_{i}=\frac{(ic\pi)^{2}}{L^{2}}\;\,,i\in\mathbb{N}\label{eq:true eig}
\end{equation}

For example, with $L=1$, $i=1$ (the lowest eigenvalue) we have $\lambda_{1}=\pi^{2}c^{2}\approx9.86c^{2}$
, and $\sqrt{\lambda_{1}}$ is the frequency (in time) of the dominating
dynamic mode Semi-discretization

It is common to do semi-discretization on (\ref{eq:wave eq}) in space,
and step the resulting ODE in time to simulate the dynamics. This
semi-discretization in space is highly related to the eigenvalue problem
(\ref{eq:eigenvalue prob}). In a sense, (\ref{eq:eigenvalue prob})
split the spatial variable from the time variable in (\ref{eq:wave eq}),
just as how the semi-discretization split the spatial variable from
the time variable. However, it is not clear 
\begin{enumerate}
\item how the discretization error from semi-discretizing in space can pollute
the overall time trajectory (dynamic behavior), \label{enu:how-the-discretization}
\item and how does this error affect the ODE time stepping error
\end{enumerate}
Here we will mainly focus on enum(\ref{enu:how-the-discretization}.).

\subsubsection{Finite difference}

The operator in (\ref{eq:eigenvalue prob}) is the negative Laplacian
scaled. For finite difference in 1D with Dirichlet BCs the Laplacian
matrix and eigenvalues are well-known. That is, using central differences
on a uiform grid the FD matrix for (\ref{eq:eigenvalue prob})
\begin{eqnarray*}
A_{h}{\bf u} & = & \lambda_{h}{\bf u}\\
A_{h} & = & \frac{-L^{2}c^{2}}{h^{2}}\left[\begin{array}{ccccc}
2 & -1\\
-1 & 2 & -1\\
 & \ddots & \ddots & \ddots\\
 &  & -1 & 2 & -1\\
 &  &  & -1 & 2
\end{array}\right]
\end{eqnarray*}
where $h$ is the interval size. The eigenvalues of this matrix is
also well-known
\begin{equation}
\lambda_{h,i}=\frac{2c^{2}L^{2}}{h^{2}}\left(1-\cos(\frac{ih\pi}{L})\right),\ h\in\mathbb{Z}^{+},\ i\in[1,\frac{L}{h}]\label{eq:eval FD}
\end{equation}

Notice that the eigenvalue from (\ref{eq:eval FD}) approach the eigenvalues
(\ref{eq:true eig}) from below as $h\rightarrow0.$ At first this
seems contradictory to what we mentioned in \ref{subsec:Theoretical-background}
and the lemma in \cite{Ciarlet:1968:NMH:2716620.2716941}. However,
this is not the case, since finite difference discretization is not
based on the weak form. 

\begin{figure}
\includegraphics[scale=0.3]{../1Dwave/FD_1D}

\caption{}
\end{figure}

\subsubsection{Finite element}

Another way to solve the eigenvalue problem (\ref{eq:eigenvalue prob})
is using finite element. First convert (\ref{eq:eigenvalue prob})
to the weak form
\begin{eqnarray*}
-c^{2}\int V\nabla^{2}Udx & =c^{2}\int\nabla U\nabla Vdx & -c^{2}\int\nabla\cdot(V\nabla U)dx
\end{eqnarray*}
and divergence theorem says the second term on the RHS is $0$ with
Dirichlet BCs, so (\ref{eq:eigenvalue prob}) is
\begin{equation}
c^{2}\int\nabla U\nabla Vdx=\lambda\int VUdx\label{eq:weak form eig}
\end{equation}
In Galerkin method we choose the same linear hat function as the function
space for $U$ and $V$. The resulting element stiffness matrix is
\[
K_{e,h}=\frac{Lc^{2}}{h}\left[\begin{array}{cc}
1 & -1\\
-1 & 1
\end{array}\right]
\]

and element mass matrix from the RHS
\[
M_{e,h}=\frac{\lambda h}{6L}\left[\begin{array}{cc}
2 & 1\\
1 & 2
\end{array}\right]
\]
After assembly and eliminate the boundary variables from Dirichlet
BCs, we have the system
\begin{eqnarray}
K_{h}{\bf u} & = & \lambda M_{h}{\bf u}\label{eq:eig FE}\\
K_{h} & = & \frac{-Lc^{2}}{h}\left[\begin{array}{ccccc}
2 & -1\\
-1 & 2 & -1\\
 & \ddots & \ddots & \ddots\\
 &  & -1 & 2 & -1\\
 &  &  & -1 & 2
\end{array}\right]\nonumber \\
M_{h} & = & \frac{h}{6L}\left[\begin{array}{ccccc}
4 & 2\\
2 & 4 & 2\\
 & \ddots & \ddots & \ddots\\
 &  & 2 & 4 & 2\\
 &  &  & 2 & 4
\end{array}\right]\nonumber 
\end{eqnarray}
The analytical solution to the generalized eigenvalue problem (\ref{eq:eig FE})
is not known yet, but it seems like it is following
\begin{equation}
\lambda_{h,i}=-\frac{2c^{2}L^{2}}{h^{2}}\left(1-\cosh(\frac{ih\pi}{L})\right),\ h\in\mathbb{Z}^{+},\ i\in[1,\frac{L}{h}].\label{eq:eig fe guess}
\end{equation}
Notice that (\ref{eq:eig fe guess}) approach (\ref{eq:true eig})
from above as $h\rightarrow0$.

\begin{figure}
\includegraphics[scale=0.3]{../1Dwave/FE_1D}

\caption{}
\end{figure}

\subsubsection{Story so far}

The simple analysis above shows that, the eigenvalues to (\ref{eq:true eig}),
which represents the dynamic property of the solution, is affected
by the mesh resolution used in the semi-discretization. The effect
is not huge numerically ($\mathcal{O}(h^{2}))$, but the resulting
dynamic simulation may look very different. This is a simplest aspect
of enum(\ref{enu:how-the-discretization}1.). A weird/interesting
discovery on the path was that changing from FD to FE reversed the
$h$-dependence of the eigenvalues. (Figure 1 is like a mirrored inverse
of Figure 2, observed by Danny)

\subsubsection{DAC fix }

DAC provides a simple to match the dynamic property of the original
continuous equation (\ref{eq:wave eq}). That is, we can scale the
eigenvalue corresponding to the dominating mode (the lowest eigenvalue).
If we do that, we can rescale the stiffness matrix by $\frac{\lambda_{1}}{\lambda_{h,1}}$.
The resulting plot for the lowest eigenvalue is in Figure 3. The difference
between the method in \cite{Chen:2017:DNC:3072959.3073669} and what
I used here is that, the rescaling factor is derived from an analytical
form of the lowest eigenvalue (the dominating mode), which is not
always avaible. To handle more complicated objects, \cite{Chen:2017:DNC:3072959.3073669}
used a vision-based technique to extract the dominating mode of motion,
as described in Section \ref{sec:Fitting-Young's-modulus}.

\begin{figure}
\includegraphics[scale=0.3]{../1Dwave/FE_1D_fix}

\caption{}
\end{figure}

\subsubsection{No free lunch}

However, there's no free lunch, and this simple scaling won't fix
other eigenvalues. For example, the second eigenvalue will be scaled
to another number, but not its continuous counterpart, as shown in
Figure 4. 

\begin{figure}
\includegraphics[scale=0.3]{../1Dwave/FE_1D_2nd_fix_wrong}

\caption{}
\end{figure}

\subsection{2D static example}

Here is another demonstration in the static setting

\begin{figure}
\includegraphics[scale=0.6]{report_images/static_tri_single}

\caption{Force required to move a single triangle element\label{fig:Force-required-to}}

\end{figure}

\begin{figure}
\includegraphics[scale=0.6]{report_images/static_tri_fine_mesh}

\caption{Force required to achieve\label{fig:Force-required-to-1} the same
deformation on a refined mesh}

\end{figure}

In Figure \ref{fig:Force-required-to} and \ref{fig:Force-required-to-1},
the yellow triangle is the rest shape. In Figure \ref{fig:Force-required-to}
we deformed the top node and have the lower 2 fixed, changing it into
the blue triangle. The red arrows are the force required to hold the
blue triangle in this deformed shape. If we refine the triangle, as
shown in Figure \ref{fig:Force-required-to-1}, and perform the same
operation, that is, move the top node while keeping the bottom two
corners fixed (all other nodes are free), the force required at each
nodes to keep the deformed triangle is much less.

\subsection{Locking/Numerical Stiffening}

The mesh-dependent stiffness is caused by the fact the function space
spanned by the bases functions is not rich enough, which is also the
same cause for the locking effect discussed in engineering literature:
when the Poisson ratio is high, it takes force up to an order of magnitude
larger to bend linear elements to the same amount of deformation on
a coarse mesh. Thus it is worth trying different ways people mitigate
locking. However, we should keep in mind that we have a different
application:
\begin{itemize}
\item In engineering, the force magnitude is usually much larger, and the
focus is usually the displacement
\item For dynamic simulation in graphics, we also care about oscillation
frequency, since collision is highly sensitive to the trajectory. 
\end{itemize}

\subsection{Discontinuous Galerkin}

Discontinuous Galerkin has been used to solve this problem in \cite{kaufmann2009flexible},
so the next step is to study this paper.

\bibliographystyle{plain}
\bibliography{report_summer_2017}

\end{document}
